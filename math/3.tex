\documentclass[utf8]{ctexart}
\pagestyle{headings}
\markright{数学竞赛第三次辅导}
\usepackage[a4paper,left=2.3cm,right=2.3cm]{geometry}
\usepackage{amsmath}
\usepackage{amssymb}
\usepackage{upgreek}
\newcommand{\tk}[1][1.5]{\underline{\mbox{\hspace{#1 cm}}}}
\newcommand{\abs}[1]{\lvert #1 \rvert}
\DeclareMathOperator\diff{d\!}
\newcommand{\e}{\mathrm{e}}
\begin{document}
\begin{enumerate}
   \item $\int^1_{\e^{-n\uppi}}\abs{[\cos(\frac{1}{x})]'}\ln\frac{1}{x}\diff x$=\tk.
   \item 设$f(x)=x,g(x)=\begin{cases}
	 \sin x,&0\leqslant x\leqslant\frac{\uppi}{2}\\
	 0,&x>\frac{\uppi}{2}
      \end{cases}$.求$F(x)=\int^x_0f(t)g(x-t)\diff t$.
   \item 设$f(x)$在$[0,\uppi]$连续,在$(0,\uppi)$可导,且$\int^\uppi_0f(x)\cos x\diff x=\int^\uppi_0f(x)\sin x\diff x=0$,证明:
      \[
	 \exists\xi\in(0,\uppi) f'(\xi)=0
      \]
   \item 设$f(x)=\int^x_{-1}t\abs{t}\diff t$,求曲线$y=f(x)$和$x$轴所围图形的面积.
   \item 证明:
      \[
	 \forall\lambda\in\mathbb{R} \int^\frac{\uppi}{2}_0\frac{1}{1+\tan^\lambda x}\diff x=\int^\frac{\uppi}{2}_0\frac{1}{1+\cot^\lambda x}\diff x=\frac{\uppi}{4}
      \]
   \item 可微函数$f(x)$在$x>0$处有定义,其反函数$g(x)$满足$\int^{f(x)}_1g(t)\diff t=\frac13(x^{\frac32}-8)$.求$f(x)$.
   \item $f(x)$在$(\-\infty,+\infty)$有界且导函数连续.$\forall x\in\mathbb{R},\abs{f(x)+f'(x)}\leqslant1$.证明$\abs{f(x)}\leqslant1$.
   \item $f(x)$在$[-L,L]$连续,$x=0$处可导,$f'(0)\neq0$.
      \begin{enumerate}
	 \item 证明:
	    \[
	       \forall 0<x<L\exists0<\theta<1 \int^x_0f(t)\diff t+\int^{-x}_0f(t)\diff t=x[f(\theta x)-f(-\theta x)]
	    \]
	 \item 求$\lim\limits_{x\rightarrow 0^+}\theta$.
      \end{enumerate}
   \item 证明:
      \[
	 \int^\uppi_0xa^{\sin x}\diff x\int^\frac{\uppi}{2}_0a^{-\cos x}\diff x\geqslant\frac{\uppi^3}{4}
      \]
   \item 证明:
      \[
	 \int^\frac{\uppi}{2}_0\frac{\sin x}{1+x^2}\diff x\leqslant\int^\frac{\uppi}{2}_0\frac{\cos x}{1+x^2}\diff x
      \]
\end{enumerate}
\end{document}
